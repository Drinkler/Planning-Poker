%!TEX root = ../dokumentation.tex
\chapter{Fazit}\label{ch:fazit}
\section{Gewonnene Erkenntnisse}\label{sec:gewonnene-erkenntnisse}
Aufgrund der Auswahl etlicher nicht in den Anforderungen enthaltenen Techniken \&{} Tools war der Lerneffekt dieser Arbeit größer als zunächst angenommen. Die Verwendung des MVC-Patterns hat einen tiefen Einblick in professionell geschriebene PHP Anwendungen gegeben. Kleinere Tools wie Bower und Composer haben interessante Wege der Verwaltung und Einhaltung von Abhängigkeiten gezeigt.\\
Das verwenden eines Server Hosting und Cloud Services Anbieters zeigte viele neue Einblicke in die Verwaltung von Servern, Cloud-Diensten und Integrationen von Drittanbieter. Die Verwendung eines CI Tools führte zu vielen neuen Erkenntnissen im Bereich der Bereitstellung und Testen einer Webanwendung.

\section{Ausblick}\label{sec:ausblick}
Um die Webanwendung zu erweitern könnte man die bereits angesprochenen Probleme beheben.
Darüber hinaus wäre es möglich, die Anwendung in einem Docker Container zu betreiben, um eine einheitliche Entwicklungsumgebung zu schaffen. Einen weiteren Mehrwert bildet sich hierbei durch die einfachere Implementierung neuer Services. Die Implementierung eines E-Mail Servers zur angemessenen Verifizierung eines Benutzeraccounts wäre somit denkbar. 
Ein weiterer Punkt wäre die vollkommene Abdeckung der Anwendung durch Unit-Tests, um stets vollste Funktionalität der Anwendung zu gewährleisten. Darüber hinaus wäre es sauberer, Ajax Aufrufe nicht in einem vorgegebenen Intervall auszuführen, sondern durch Benachrichtigung des Clients durch einen Webhook.