%!TEX root = ../dokumentation.tex
\chapter{Entwicklung}\label{ch:entwicklung}
Für die Entwicklung der Software wurde von Florian die IDE Visual Studio Code verwendet und von Luca die IDE Jetbrains PhpStorm.
Zum Verwalten der Datenbank wurde die Software Jetbrains DataGrip.

\section{Toolchain}\label{sec:toolchain}

\section{Bibliotheken/Frameworks}\label{sec:bibliotheken/frameworks}
\subsection{Bower}\label{subsec:bower}
\subsection{Composer}\label{subsec:composer}
\subsection{phpunit}\label{subsec:phpunit}

\section{MVC}\label{sec:mvc}
\subsection{.htaccess}\label{subsec:.htaccess}
\subsection{Model}\label{subsec:model}
\subsection{View}\label{subsec:view}
\subsection{Controller}\label{subsec:controller}

\section{Funktionalität}\label{sec:funktionalität}

\subsection{Login/Register}\label{subsec:login/register}
passwort hashing
confirm ohne email wegen email server in aws.
logik der registrierung und anmeldung

\subsection{Lobby}\label{subsec:lobby}

\subsubsection{Scrum Master}
\subsubsection{Spieler}