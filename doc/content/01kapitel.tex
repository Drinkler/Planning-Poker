%!TEX root = ../dokumentation.tex

\chapter{Einleitung}\label{ch:einleitung}
\section{Projektteam}\label{sec:projektteam}
Das Projektteam bestand aus folgenden Mitgliedern:
\begin{itemize}
	\item Florian Drinkler - 6653948 - inf18149@lehre.dhbw-stuttgart.de
	\item Luca Stanger - 7474265 - inf18244@lehre.dhbw-stuttgart.de
\end{itemize}
\section{Aufgabenverteilung}\label{sec:aufgabenverteilung}
Für Frontend Entwicklung sowie dem Entwurf und der Einhaltung des MVC Patterns war Luca Stanger zuständig. Die Pflege von Backend, Deployment, Datenbankverbindungen und SQL Abfragen fiel in Florian Drinkler's Aufgabengebiet. 
\section{Programmablauf}\label{sec:programmablauf}
Um dem Spielbetrieb entweder als Mitspieler oder als Scrum Master beitreten zu können, muss ein Benutzeraccount erstellt werden. Dies funktioniert über die Fläche \emph{Sign Up}. Nach erfolgreicher Erstellung des Accounts muss über einen Button die E-Mail bestätigt werden. Im Anschluss an die Bestätigung kann Sich der Benutzer anmelden. Nimmt der Nutzer innerhalb des Teams die Rolle des Scrum-Mastes ein, erstellt dieser eine neue Lobby und wählt ein Kartenset aus. Anschließend muss der Lobby beigetreten und den restlichen Teammitgliedern die Lobby ID mitgeteilt werden. Nachdem alle Teammitglieder beigetreten sind, kann der Scrum Master ein öffentliches GitHub-Repository in die dafür vorgesehene Eingabemaster einlesen. Eine dafür angefertigte Schnittstelle ließt alle im Projekt vorhandenen Issues aus und stellt diese dem Scrum-Master zur Verfügung. Mit einem Klick auf die Schaltfläche \emph{aktivieren} kann die User-Story den restlichen Mitgliedern angezeigt werden. Alle Mitglieder der Lobby können nach Aktivieren einer Story ihre Stimme abgeben. Der Scrum Master erhält eine Übersicht über alle abgegebenen Stimmen und kann den Durchschnittswert aller Karten auslesen.